\subsection*{}
\vskip -1cm
\noindent 1.证明自由空间格林函数的偏导数关系:
$$\frac{\partial G_{0}}{\partial y_{i}}=
\frac{x_{i}-y_{i}}{r}\left[\frac{1}{4 \pi r c_{0}} \frac{\partial}{\partial \tau} \delta\left(t-\tau-r / c_{0}\right)+
\frac{\delta\left(t-\tau-r / c_{0}\right)}{4 \pi r^{2}}\right].$$

\noindent 已知自由空间格林函数:
\begin{equation}
    G_{0}=\frac{1}{4 \pi r} \delta\left(t-\tau-\frac{r}{c_{0}}\right)     
\end{equation}
对$r$求偏导,得:
\begin{equation}
    \begin{aligned}
        \label{eq:parital r}
        \frac{\partial G_{0}}{\partial r} &= -\frac{1}{4 \pi r^{2}} \delta\left(t-\tau-\frac{r}{c_{0}}\right)+\frac{1}{4 \pi r} \frac{\partial \delta\left(t-\tau -\frac{r}{c_{0}}\right)}{\partial r} \\
        &= -\frac{1}{4 \pi r^{2}} \delta\left(t-\tau-\frac{r}{c_{0}}\right)+\frac{1}{4 \pi r} \frac{\partial \delta\left(t-\tau -\frac{r}{c_{0}}\right)}{\partial \tau} \frac{\partial \tau}{\partial r}   
    \end{aligned}
\end{equation}
由$ \tau $与$ r $的关系式$ \tau = t - \frac{r}{c_{0}} $可得:
\begin{equation}
    \frac{\partial \tau}{\partial r} = - \frac{1}{c_{0}}
\end{equation} 
代入式\eqref{eq:parital r},得
\begin{equation}
    \label{eq:parital r final}
    \frac{\partial G_{0}}{\partial r} = -\frac{1}{4 \pi r^{2}} \delta\left(t-\tau-\frac{r}{c_{0}}\right)-\frac{1}{4 \pi r c_{0}} \frac{\partial \delta\left(t-\tau -\frac{r}{c_{0}}\right)}{\partial \tau}
\end{equation}
根据$r$对$y_{i}$的偏导数:
\begin{equation}
    \label{eq:r partial yi}
    \begin{aligned}
        \frac{\partial r}{\partial y_{i}} &=  \frac{\partial \sqrt{\sum (x_{i} - y_{i})^2}}{\partial y_{i}}
        &= - \frac{x_{i} - y_{i}}{r}
    \end{aligned}
\end{equation}
结合式\eqref{eq:parital r final},\eqref{eq:r partial yi},得:
\begin{equation}
    \label{eq:G0 partial yi}
    \begin{aligned}
        \frac{\partial G_{0}}{\partial y_{i}} &= \frac{\partial G_{0}}{\partial r} \frac{\partial r}{\partial y_{i}} \\
        &= \frac{x_{i}-y_{i}}{r}\left[\frac{1}{4 \pi r c_{0}} \frac{\partial}{\partial \tau} \delta\left(t-\tau-r / c_{0}\right)+
            \frac{\delta\left(t-\tau-r / c_{0}\right)}{4 \pi r^{2}}\right]
    \end{aligned}
\end{equation}
原式得证。

\clearpage

\noindent 2.利用上述自由空间格林函数的偏导数关系式证明
$$
p^{\prime}(\mathbf{x}, t)=-\int_{-\infty}^{+\infty} \int_{S} \rho_{0} \frac{\partial u_{n}(\mathbf{y}, \tau)}{\partial \tau} G(\mathbf{x}, \mathbf{y}, t-\tau) \mathrm{d} S \mathrm{~d} \tau-\int_{-\infty}^{+\infty} \int_{S} p^{\prime}(\mathbf{y}, \tau) \frac{\partial G(\mathbf{x}, \mathbf{y}, t-\tau)}{\partial y_{i}} n_{i} \mathrm{~d} S \mathrm{~d} \tau
$$
可以改写为
$$
p^{\prime}(\mathbf{x}, t)=-\int_{S}\left[\rho_{0} \frac{\partial u_{n}}{\partial \tau}\right]_{\tau} \frac{\mathrm{d} S(\mathbf{y})}{4 \pi r}-\int_{S}\left[\frac{\partial p^{\prime}}{\partial \tau} n_{i}+\frac{p^{\prime} n_{i} c_{0}}{r}\right]_{\tau} \frac{\left(x_{i}-y_{i}\right) \mathrm{d} S(\mathbf{y})}{4 \pi r^{2} c_{0}}.
$$
原式右侧第一项代入自由格林函数,并对$\tau$求积分:
\begin{equation}
    \label{eq:right first final}
    \begin{aligned}
        \mbox{右侧第一项} &= -\int_{-\infty}^{+\infty} \int_{S} \rho_{0} \frac{\partial u_{n}(\mathbf{y}, \tau)}{\partial \tau} G_{0}(\mathbf{x}, \mathbf{y}, t-\tau) \mathrm{d} S \mathrm{~d} \tau \\
        &= -\int_{s} \int_{-\infty}^{+\infty} \rho_{0} \frac{\partial u_{n}(\mathbf{y}, \tau)}{\partial \tau} \frac{1}{4 \pi r} \delta(\mathbf{x}, \mathbf{y}, t-\tau) \mathrm{d} \tau \mathrm{~d} s \\
        &= -\int_{S}\left[\rho_{0} \frac{\partial u_{n}}{\partial \tau}\right]_{\tau} \frac{\mathrm{d} S(\mathbf{y})}{4 \pi r}
    \end{aligned}
\end{equation}
原式右侧第二项代入自由格林函数偏导数关系式(式\eqref{eq:G0 partial yi}):
\begin{equation}
    \label{eq:second}
    \begin{aligned}
        \mbox{右侧第二项} &= -\int_{-\infty}^{+\infty} \int_{S} p^{\prime}(\mathbf{y}, \tau) 
        \frac{x_{i}-y_{i}}{r}\left[\frac{1}{4 \pi r c_{0}} \frac{\partial}{\partial \tau} \delta\left(t-\tau-\frac{r}{c_{0}} \right)+
            \frac{\delta\left(t-\tau-\frac{r}{c_{0}}\right)}{4 \pi r^{2}}\right]
        n_{i} \mathrm{~d} S \mathrm{~d} \tau \\
        &= -\int_S \left[ \int_{-\infty}^{+\infty} p^{\prime}(\mathbf{y}, \tau) \frac{\partial}{\partial \tau} \delta\left(t-\tau-\frac{r}{c_{0}} \right) n_i \mathrm{d} \tau\right]  \frac{\left( x_{i}-y_{i} \right) \mathrm{d} S}{4 \pi r^2 c_{0}} \\
        &{~~~}-\int_S \left[ \int_{-\infty}^{+\infty} p^{\prime}(\mathbf{y}, \tau) \frac{c_0}{r} \delta\left(t-\tau-\frac{r}{c_{0}} \right) n_i \mathrm{d} \tau\right]  \frac{\left( x_{i}-y_{i} \right) \mathrm{d} S}{4 \pi r^2 c_{0}}
    \end{aligned}
\end{equation}
其中:
\begin{equation}
    \label{eq:int p multi delt partial tau}
    \begin{aligned}
        \int_{-\infty}^{+\infty} p^{\prime}(\mathbf{y}, \tau) \frac{\partial}{\partial \tau} \delta\left(t-\tau-\frac{r}{c_{0}} \right) n_i \mathrm{d} \tau &=  \left.-p^{\prime}(\mathbf{y}, \tau)\delta(t-\tau-\frac{r}{c_0}) n_i \right|_{\tau = -\infty}^{\tau = \infty} \\
        &{~~~} - \int_{-\infty}^{+\infty} -\frac{\partial p^{\prime}(\mathbf{y}, \tau)}{\partial \tau}\delta(t-\tau-\frac{r}{c_0}) \mathrm{d}\tau
    \end{aligned}
\end{equation}
根据$t$与$\tau$的因果关系,有:
\begin{equation}
    \left.-p^{\prime}(\mathbf{y}, \tau)\delta(t-\tau-\frac{r}{c_0}) n_i \right|_{\tau = -\infty}^{\tau = \infty} = 0
\end{equation}
因此有:
\begin{equation}
    \label{eq:result first}
    \begin{aligned}
        \int_{-\infty}^{+\infty} p^{\prime}(\mathbf{y}, \tau) \frac{\partial}{\partial \tau} \delta\left(t-\tau-\frac{r}{c_{0}} \right) n_i \mathrm{d} \tau &=
        - \int_{-\infty}^{+\infty} -\frac{\partial p^{\prime}(\mathbf{y}, \tau)}{\partial \tau}\delta(t-\tau-\frac{r}{c_0}) \mathrm{d}\tau \\
        &= \left[ \frac{\partial p^{\prime}}{\partial \tau} n_i \right]_{\tau}
    \end{aligned}
\end{equation}
同时,式\eqref{eq:second}中:
\begin{equation}
    \label{eq:result second}
    \int_{-\infty}^{+\infty} p^{\prime}(\mathbf{y}, \tau) \frac{c_0}{r} \delta\left(t-\tau-\frac{r}{c_{0}} \right) n_i \mathrm{d} \tau
    = \left[ \frac{p^{\prime} n_i c_0}{r}\right]_{\tau}
\end{equation}
将式\eqref{eq:result first},\eqref{eq:result second}代入式\eqref{eq:second},得:
\begin{equation}
    \label{eq:right second final}
    \begin{aligned}
        \mbox{右侧第二项} &= -\int_S  \left[ \frac{\partial p^{\prime}}{\partial \tau} n_i \right]_{\tau} \frac{\left( x_{i}-y_{i} \right) \mathrm{d} S}{4 \pi r^2 c_{0}} 
        -\int_S  \left[ \frac{p^{\prime} n_i c_0}{r}\right]_{\tau} \frac{\left( x_{i}-y_{i} \right) \mathrm{d} S}{4 \pi r^2 c_{0}} \\
        &= -\int_S  \left[ \frac{\partial p^{\prime}}{\partial \tau} n_i + \frac{p^{\prime} n_i c_0}{r} \right]_{\tau} \frac{\left( x_{i}-y_{i} \right) \mathrm{d} S}{4 \pi r^2 c_{0}}
    \end{aligned}
\end{equation}
结合式\eqref{eq:right first final},\eqref{eq:right second final},得:
\begin{equation}
    p^{\prime}(\mathbf{x}, t)=-\int_{S}\left[\rho_{0} \frac{\partial u_{n}}{\partial \tau}\right]_{\tau} \frac{\mathrm{d} S(\mathbf{y})}{4 \pi r}-\int_{S}\left[\frac{\partial p^{\prime}}{\partial \tau} n_{i}+\frac{p^{\prime} n_{i} c_{0}}{r}\right]_{\tau} \frac{\left(x_{i}-y_{i}\right) \mathrm{d} S(\mathbf{y})}{4 \pi r^{2} c_{0}}
\end{equation}
原式得证。





