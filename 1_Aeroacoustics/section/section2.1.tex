\section*{Section2.1}

\begin{enumerate}
    \item Lighthill声比拟方程能直接应用于高Ma流动诱发的气动噪声问题吗? \\
        不能。
        (1)Lighthill声比拟方程假设空气介质是均匀静止的,但该条件不能适用于高马赫数流动;
        (2)Lighthill声比拟方程没有考虑能量输运作用;
        (3)Lighthill声比拟方程仅适用于弱可压缩流动,不适用于高马赫数下的强可压缩流动。


    \item 从Lighthill声比拟方程出发,详细证明方程的时域积分解为
    % \begin{equation*}
    %     p^{\prime}(\boldsymbol{x}, t)=
    %     \frac{1}{4 \pi} \frac{\partial^{2}}{\partial x_{i} \partial x_{j}} \int_{V} \frac{T_{i j}\left(\boldsymbol{y}, t-r / c_{0}\right)}{r} \mathrm{~d}^{3} \boldsymbol{y}
    % \end{equation*}
    % 已知声比拟方程:
    % \begin{equation}
    %     \frac{\partial^{2} \rho^{\prime}}{\partial t^{2}}-c_{0}^{2} \frac{\partial^{2} \rho^{\prime}}{\partial x_{i}^{2}}
    %     = \frac{\partial^{2} T_{i j}}{\partial x_{i} \partial x_{j}}
    % \end{equation}
    % 代入$\rho^{\prime}$与$p^{\prime}$的关系式:
    % \begin{equation}
    %     \rho^{\prime} = \frac{1}{c_{0}^{2}} p^{\prime}
    % \end{equation}
    % 声比拟方程可改写为:
    % \begin{equation}
    %     \label{eq:sheng bi ni fangcheng}
    %     \frac{1}{c_{0}^{2}} \frac{\partial^{2} p^{\prime}(\mathbf{y},\tau)}{\partial \tau^{2}}- \frac{\partial^{2} p^{\prime}(\mathbf{y},\tau)}{\partial y_{i}^{2}}
    %     = \frac{\partial^{2} T_{i j}}{\partial y_{i} \partial y_{j}}
    % \end{equation}
    % 已知格林函数满足关系式:
    % \begin{equation}
    %     \label{eq:ge lin han shu man zu guan xi shi}
    %     \frac{1}{c_{0}^{2}} \frac{\partial^{2} G(\mathbf{x},\mathbf{y},\tau)}{\partial \tau^{2}}-\frac{\partial^{2} G(\mathbf{x},\mathbf{y},\tau)}{\partial y_{i}^{2}} 
    %     = \delta(\mathbf{x}-\mathbf{y}) \delta(t-\tau)
    % \end{equation}
    % 将式\eqref{eq:sheng bi ni fangcheng}乘上$G(\mathbf{x},\mathbf{y},\tau)$,
    % 式\eqref{eq:ge lin han shu man zu guan xi shi}乘上$p^{\prime}(\mathbf{y},\tau)$,
    % 并相减,得:
    % \begin{equation}
    %     \begin{aligned}
    %         \delta(\mathbf{x}-\mathbf{y}) \delta(t-\tau) p^{\prime}(\mathbf{y},\tau) 
    %         =& \frac{1}{c_{0}^{2}} \left[ p^{\prime} \frac{\partial^{2} G}{\partial \tau^{2}} - G \frac{\partial^{2} p^{\prime}}{\partial \tau^{2}} \right] 
    %          - \left[ p^{\prime} \frac{\partial^{2} G}{\partial y_{i}^{2}} - G \frac{\partial^{2}  p^{\prime}}{\partial y_{i}^{2}} \right] \\
    %          & + G \frac{\partial^{2} T_{i j}}{\partial y_{i} \partial y_{j}}
    %     \end{aligned}
    % \end{equation}
    % 对上式两边对时间和体积进行积分,得:
    % \begin{equation}
    %     \begin{aligned}
    %         p^{\prime}(\mathbf{x},\tau) = 
    %         & \int_{V} \int_{-\infty}^{\infty} \frac{1}{c_{0}^{2}} \left[ p^{\prime} \frac{\partial^{2} G}{\partial \tau^{2}} - G \frac{\partial^{2} p^{\prime}}{\partial \tau^{2}} \right] \mathrm{d} \tau \mathrm{d} V \\
    %         & - \int_{-\infty}^{\infty} \int_{V} \left[ p^{\prime} \frac{\partial^{2} G}{\partial y_{i}^{2}} - G \frac{\partial^{2}  p^{\prime}}{\partial y_{i}^{2}} \right] \mathrm{d} V \mathrm{d} \tau \\
    %         & + \int_{-\infty}^{\infty} \int_{V} G \frac{\partial^{2} T_{i j}}{\partial y_{i} \partial y_{j}} \mathrm{d} V \mathrm{d} \tau
    %     \end{aligned}
    % \end{equation}
    % 根据$t$与$\tau$的因果关系,有:
    % \begin{equation}
    %     \int_{V} \int_{-\infty}^{\infty} \frac{1}{c_{0}^{2}} \left[ p^{\prime} \frac{\partial^{2} G}{\partial \tau^{2}} - G \frac{\partial^{2} p^{\prime}}{\partial \tau^{2}} \right] \mathrm{d} \tau \mathrm{d} V = 0
    % \end{equation}
    % 因此有:
    % \begin{equation}
    %     \begin{aligned}
    %         p^{\prime}(\mathbf{x},\tau) = 
    %         & - \int_{-\infty}^{\infty} \int_{V} \left[ p^{\prime} \frac{\partial^{2} G}{\partial y_{i}^{2}} - G \frac{\partial^{2}  p^{\prime}}{\partial y_{i}^{2}} \right] \mathrm{d} V \mathrm{d} \tau \\
    %         & + \int_{-\infty}^{\infty} \int_{V} G \frac{\partial^{2} T_{i j}}{\partial y_{i} \partial y_{j}} \mathrm{d} V \mathrm{d} \tau
    %     \end{aligned}
    % \end{equation}
    
    根据声比拟方程,有:
    \begin{equation}
        \begin{aligned}
            p^{\prime}(\mathbf{x},\tau) 
            = \int_{-\infty}^{\infty} \int_{V} G \frac{\partial^{2} T_{i j}(\mathbf{y},\tau)}{\partial y_{i} \partial y_{j}} \mathrm{d} V \mathrm{d} \tau
        \end{aligned}
    \end{equation}
    根据分步积分,有:
    \begin{equation}
        G \frac{\partial^{2} T_{i j}}{\partial y_{i} \partial y_{j}}=T_{i j} \frac{\partial^{2} G}{\partial y_{i} \partial y_{j}}+\frac{\partial}{\partial y_{i}}\left(G \frac{\partial T_{i j}}{\partial y_{j}}\right)-\frac{\partial}{\partial y_{j}}\left(T_{i j} \frac{\partial G}{\partial y_{i}}\right)
    \end{equation}
    因此有:
    \begin{equation}
        \begin{aligned}
            \int_{-\infty}^{\infty} \int_{V} G \frac{\partial^{2} T_{i j}}{\partial y_{i} \partial y_{j}} \mathrm{d} V \mathrm{d} \tau =
            & \int_{-\infty}^{\infty} \int_{V} T_{i j} \frac{\partial^{2} G}{\partial y_{i} \partial y_{j}} \mathrm{d} V \mathrm{d} \tau \\
            & + \int_{-\infty}^{\infty} \int_{V} \frac{\partial}{\partial y_{i}}\left(G \frac{\partial T_{i j}}{\partial y_{j}}\right) \mathrm{d} V \mathrm{d} \tau \\
            & - \int_{-\infty}^{\infty} \int_{V} \frac{\partial}{\partial y_{j}}\left(T_{i j} \frac{\partial G}{\partial y_{i}}\right) \mathrm{d} V \mathrm{d} \tau
        \end{aligned}
    \end{equation}
    注意到$T_{i j} = T_{j i}$,因此有:
    \begin{equation}
        \begin{aligned}
            \int_{-\infty}^{\infty} \int_{V} G \frac{\partial^{2} T_{i j}}{\partial y_{i} \partial y_{j}} \mathrm{d} V \mathrm{d} \tau =
            & \int_{-\infty}^{\infty} \int_{V} T_{i j} \frac{\partial^{2} G}{\partial y_{i} \partial y_{j}} \mathrm{d} V \mathrm{d} \tau \\
            & + \int_{-\infty}^{\infty} \int_{V} \frac{\partial}{\partial y_{j}}\left[G \frac{\partial T_{i j}}{\partial y_{i}}
              - T_{i j} \frac{\partial G}{\partial y_{i}}\right] \mathrm{d} V \mathrm{d} \tau
        \end{aligned}
    \end{equation}
    应用高斯散度定理,有:
    \begin{equation}
        \begin{aligned}
            \int_{-\infty}^{\infty} \int_{V} G \frac{\partial^{2} T_{i j}}{\partial y_{i} \partial y_{j}} \mathrm{d} V \mathrm{d} \tau =
            & \int_{-\infty}^{\infty} \int_{V} T_{i j} \frac{\partial^{2} G}{\partial y_{i} \partial y_{j}} \mathrm{d} V \mathrm{d} \tau \\
            & + \int_{-\infty}^{\infty} \int_{S} \left[G \frac{\partial T_{i j}}{\partial y_{i}}
              - T_{i j} \frac{\partial G}{\partial y_{i}}\right] n_{i} \mathrm{d} S \mathrm{d} \tau
        \end{aligned}
    \end{equation}
    对于Lighthill声比拟方程,$S$为无穷大,因此有:
    \begin{equation}
        \int_{-\infty}^{\infty} \int_{S} \left[G \frac{\partial T_{i j}}{\partial y_{i}}
        - T_{i j} \frac{\partial G}{\partial y_{i}}\right] n_{i} \mathrm{d} S \mathrm{d} \tau
        = 0
    \end{equation}
    因此:
    \begin{equation}
        \begin{aligned}
            p^{\prime}(\mathbf{x},\tau) 
            &= \int_{-\infty}^{\infty} \int_{V} G \frac{\partial^{2} T_{i j}(\mathbf{y},\tau)}{\partial y_{i} \partial y_{j}} \mathrm{d} V \mathrm{d} \tau \\
            &= \int_{-\infty}^{\infty} \int_{V} T_{i j}(\mathbf{y},\tau) \frac{\partial^{2} G}{\partial y_{i} \partial y_{j}} \mathrm{d} V \mathrm{d} \tau 
        \end{aligned}
    \end{equation}
    代入自由空间格林函数$G_{0}$,有:
    \begin{equation}
        \begin{aligned}
            p^{\prime}(\mathbf{x},\tau) 
            &= \int_{-\infty}^{\infty} \int_{V} T_{i j}(\mathbf{y},\tau) \frac{\partial^{2} G_{0}(\mathbf{x},\mathbf{y},t-\tau)}{\partial y_{i} \partial y_{j}} \mathrm{d} V \mathrm{d} \tau 
        \end{aligned}
    \end{equation}
    自由空间格林函数$G_{0}$满足:
    \begin{equation}
        \frac{\partial^{2} G_{0}}{\partial y_{i} \partial y_{j}} 
        = \frac{\partial^{2} G_{0}}{\partial x_{i} \partial x_{j}} 
    \end{equation}
    因此有:
    \begin{equation}
        \begin{aligned}
            p^{\prime}(\mathbf{x},\tau) 
            &= \int_{-\infty}^{\infty} \int_{V} T_{i j}(\mathbf{y},\tau) \frac{\partial^{2} G_{0}(\mathbf{x},\mathbf{y},t-\tau)}{\partial x_{i} \partial x_{j}} \mathrm{d}^{3} \mathbf{y} \mathrm{~d} \tau \\
            &= \frac{\partial^{2}}{\partial x_{i} \partial x_{j}} \int_{V} \int_{-\infty}^{\infty} T_{i j}(\mathbf{y},\tau)  G_{0}(\mathbf{x},\mathbf{y},t-\tau) \mathrm{d} \tau \mathrm{~d}^{3} \mathbf{y} \\
            &= \frac{\partial^{2}}{\partial x_{i} \partial x_{j}} \int_{V} \int_{-\infty}^{\infty} T_{i j}(\mathbf{y},\tau)  \frac{\delta(t - \tau - r/c_{0})}{4 \pi r} \mathrm{d} \tau \mathrm{~d}^{3} \mathbf{y} \\
            &= \frac{1}{4 \pi} \frac{\partial^{2}}{\partial x_{i} \partial x_{j}} \int_{V} \frac{T_{i j}\left(\mathbf{y}, t-r / c_{0}\right)}{r} \mathrm{~d}^{3} \mathbf{y}
        \end{aligned}
    \end{equation}
    原式得证。
\end{enumerate}

\clearpage