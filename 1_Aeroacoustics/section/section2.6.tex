\section*{Section2.6}

\begin{enumerate}
    \item 对均匀静止介质中以攻数  \(M_{i} \) 运动的声源, 证明 
        \[ \frac{\partial M_{r}}{\partial \tau}
        = \frac{1}{r}\left\{r_{i} \frac{\partial M_{i}}{\partial \tau}+c_{0}\left(M_{r}^{2}-M^{2}\right)\right\}
        , M=\sqrt{M_{1}^{2}+M_{2}^{2}+M_{3}^{2}} \]

        已知:
        \begin{equation}
            M_{r}=\frac{r_{i} M_{i}}{r}
        \end{equation}
        根据微分公式,有:
        \begin{equation}
            \label{eq:weifeng}
            \begin{aligned}
                \frac{\partial M_{r}}{\partial \tau}
                &= \frac{\partial}{\partial \tau} (\frac{r_{i} M_{i}}{r}) \\
                &= \frac{r_{i}}{r} \frac{\partial M_{i}}{\partial \tau}
                + \frac{M_{i}}{r} \frac{\partial r_{i}}{\partial \tau}
                - \frac{r_{i} M_{i}}{r^{2}} \frac{\partial r}{\partial \tau}
            \end{aligned}
        \end{equation}
        其中:
        \begin{equation}
            \frac{M_{i}}{r} \frac{\partial r_{i}}{\partial \tau}
            = \frac{M_{i}}{r} (-v_{i})
            = -\frac{M_{i}^{2} c_{0}}{r}
        \end{equation}
        \begin{equation}
            \frac{r_{i} M_{i}}{r^{2}} \frac{\partial r}{\partial \tau}
            = \frac{M_{r}}{r} ( -M_{r} c_{0} )
            = -\frac{M_{r}^{2} c_{0}}{r}
        \end{equation}
        带入式\eqref{eq:weifeng},得:
        \begin{equation}
            \begin{aligned}
                \frac{\partial M_{r}}{\partial \tau}
                &= \frac{r_{i}}{r} \frac{\partial M_{i}}{\partial \tau}
                + \frac{M_{i}}{r} \frac{\partial r_{i}}{\partial \tau}
                - \frac{r_{i} M_{i}}{r^{2}} \frac{\partial r}{\partial \tau} \\
                &= \frac{r_{i}}{r} \frac{\partial M_{i}}{\partial \tau}
                -\frac{M_{i}^{2} c_{0}}{r}
                +\frac{M_{r}^{2} c_{0}}{r} \\
                &= \frac{1}{r}\left\{r_{i} \frac{\partial M_{i}}{\partial \tau}
                + c_{0}\left(M_{r}^{2}-M_{i}^{2}\right)\right\} \\
                &= \frac{1}{r}\left\{r_{i} \frac{\partial M_{i}}{\partial \tau}
                + c_{0}\left(M_{r}^{2}-M^{2}\right)\right\}
                , M=\sqrt{M_{1}^{2}+M_{2}^{2}+M_{3}^{2}}
            \end{aligned}
        \end{equation}

    \item 证明偶极子噪声的积分表达式
        \begin{equation*}
            \begin{aligned}
                \pi p_{D}(\mathbf{x}, t)
                =& \int_{S}\left[\frac{r_{i}}{r^{2} c_{0}\left(1-M_{r}\right)^{2}}\left\{\frac{\partial F_{i}}{\partial \tau}+\frac{F_{i}}{1-M_{r}}\left(\frac{r_{j}}{r} \frac{\partial M_{j}}{\partial \tau}\right)\right\}\right] \mathrm{d}^{2} \mathbf{y} \\
                &+\int_{S}\left[\frac{1}{r^{2}\left(1-M_{r}\right)^{2}}\left\{\frac{F_{i} r_{i}}{r} \frac{1-M^{2}}{1-M_{r}}-F_{i} M_{i}\right\}\right] \mathrm{d}^{2} \mathbf{y}
            \end{aligned}
        \end{equation*}
    已知:
    \begin{equation}
        \begin{aligned}
            \pi p_{D}(\mathbf{x}, t)
            = -\frac{\partial}{\partial x_{i}} \int_{S}\left[\frac{F_{i}}{r\left(1-M_{r}\right)}\right] \mathrm{d}^{2} \mathbf{y}
        \end{aligned}
    \end{equation}
    根据微分公式,有:
    \begin{equation}
        \frac{\partial}{\partial x_{i}} \left[\frac{F_{i}}{r\left(1-M_{r}\right)}\right]
        = \left[\frac{\partial}{\partial x_{i}}\left\{\frac{F_{i}}{r\left(1-M_{r}\right)}\right\}+\left[\frac{\partial \tau}{\partial x_{i}}\right.\right.  \left.\frac{\partial}{\partial \tau}\left\{\frac{F_{i}}{r\left(1-M_{r}\right)}\right\}\right]
    \end{equation}
    其中:
    \begin{equation}
        \begin{aligned}
            \frac{\partial}{\partial x_{i}}\left\{\frac{F_{i}}{r\left(1-M_{r}\right)}\right\}
            &= - \frac{F_{i}}{r^{2}\left(1-M_{r}\right)^{2}} \frac{\partial r\left(1-M_{r}\right)}{\partial x_{i}} \\
            &= - \frac{F_{i}}{r^{2}\left(1-M_{r}\right)^{2}} \left[ \frac{\partial r }{\partial x_{i}} - \frac{\partial rM_{r}}{\partial x_{i}} \right] \\
        \end{aligned}
    \end{equation}
    又因为:
    \begin{gather}
        \frac{\partial r }{\partial x_{i}} = \frac{r_{i}}{r} \\
        \frac{\partial rM_{r}}{\partial x_{i} } = \frac{\partial r_{i}M_{i}}{\partial x_{i} } = M_{i} \frac{\partial r_{i}}{\partial x_{i} } = M_{i}
    \end{gather}
    因此有:
    \begin{equation}
        \frac{\partial}{\partial x_{i}}\left\{\frac{F_{i}}{r\left(1-M_{r}\right)}\right\}
        = - \frac{F_{i}}{r^{2}\left(1-M_{r}\right)^{2}} \left( \frac{r_{i}}{r} - M_{i} \right)
    \end{equation}
    同时,对于:
    \begin{equation}
        \frac{\partial \tau}{\partial x_{i}} \frac{\partial}{\partial \tau}\left\{\frac{F_{i}}{r\left(1-M_{r}\right)}\right\}
    \end{equation}
    其中,
    \begin{equation}
        \frac{\partial \tau}{\partial x_{i}}
        = \frac{\partial \tau}{\partial g} \frac{\partial g}{\partial x_{i}}
        = \frac{1}{1-M_{r}} \frac{r_{i}}{r c_{0}}
        = \frac{r_{i}}{r c_{0} (1-M_{r})}
    \end{equation}
    \begin{equation}
        \begin{aligned}
            \frac{\partial}{\partial \tau}\left\{\frac{F_{i}}{r\left(1-M_{r}\right)}\right\} 
            =& F_{i} \frac{\partial}{\partial \tau}\left\{\frac{1}{r\left(1-M_{r}\right)}\right\} + \frac{1}{r\left(1-M_{r}\right)} \frac{\partial F_{i}}{\partial \tau} \\
            =& - \frac{F_{i}}{r^{2}\left(1-M_{r}\right)^{2}} \left\{ (1-M_{r}) \frac{\partial r}{\partial \tau} - r \frac{\partial M_{r}}{\partial \tau}\right\} \\
             & +\frac{1}{r\left(1-M_{r}\right)} \frac{\partial F_{i}}{\partial \tau} \\
            =& \frac{F_{i}}{r^{2}\left(1-M_{r}\right)^{2}} \left\{ (1-M_{r}) c_{0} M_{r} + r \frac{\partial M_{r}}{\partial \tau}\right\} \\
            & +\frac{1}{r\left(1-M_{r}\right)} \frac{\partial F_{i}}{\partial \tau} \\
        \end{aligned}
    \end{equation}
    综上,
    \begin{equation}
        \begin{aligned}
            \pi p_{D}(\mathbf{x}, t)
            =& -\frac{\partial}{\partial x_{i}} \int_{S}\left[\frac{F_{i}}{r\left(1-M_{r}\right)}\right] \mathrm{d}^{2} \mathbf{y} \\
            =& - \int_{S} \left[\frac{\partial}{\partial x_{i}}\left\{\frac{F_{i}}{r\left(1-M_{r}\right)}\right\}+\left[\frac{\partial \tau}{\partial x_{i}}\right.\right.  \left.\frac{\partial}{\partial \tau}\left\{\frac{F_{i}}{r\left(1-M_{r}\right)}\right\}\right] \mathrm{d}^{2} \mathbf{y} \\
            =& \int_{S}\left[\frac{r_{i}}{r^{2} c_{0}\left(1-M_{r}\right)^{2}}\left\{\frac{\partial F_{i}}{\partial \tau}+\frac{F_{i}}{1-M_{r}}\left(\frac{r_{j}}{r} \frac{\partial M_{j}}{\partial \tau}\right)\right\}\right] \mathrm{d}^{2} \mathbf{y} \\
            &+\int_{S}\left[\frac{1}{r^{2}\left(1-M_{r}\right)^{2}}\left\{\frac{F_{i} r_{i}}{r} \frac{1-M^{2}}{1-M_{r}}-F_{i} M_{i}\right\}\right] \mathrm{d}^{2} \mathbf{y}
        \end{aligned}
    \end{equation}
    原式得证。
\end{enumerate}

\clearpage