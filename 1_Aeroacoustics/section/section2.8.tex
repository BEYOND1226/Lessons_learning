\section*{Section2.8}

\begin{enumerate}
    \item 假设亚声速均匀流沿 \( x_{1} \) 轴正向运动, 
    在 \( \boldsymbol{y} \) 点有一静止声源辐射声波, 
    如果已知观察点 \( \boldsymbol{x} \) 的时间 \(t\),
    如何确定延迟时间 \( \tau \) ?
    
    
    \item  假设均匀静止介质中有一点源以恒定速度 \(v\) (亚声速) 沿 \( x_{1} \) 轴正向运动,
    其初始位置为 \( \boldsymbol{y}_{0} \), 
    如果已知观察点\( \boldsymbol{x} \)的时间\(t\), 如何确定延迟时间\( \tau \)?
    
    
    \item 均匀静止介质中, 一强度为\( q(t) \)的点源以恒定速度 \( \boldsymbol{v} \) 亚声速直线运动, 
    且 \( t=0 \) 时刻恰好经 过坐标原点, 辐射声场的速度势函数 \( \phi(\boldsymbol{x}, t) \) 满足方程
    \( \frac{1}{c_{0}^{2}} \frac{\partial^{2} \phi}{\partial t^{2}}-\nabla^{2} \phi=q(t) \delta(\boldsymbol{x}-\boldsymbol{v} t) \), 
    证明
    \[
    \phi(\boldsymbol{x}, t)=\frac{q\left(t-R / c_{0}\right)}{4 \pi R(1-M \cos \theta)}, \quad M=\frac{|\boldsymbol{v}|}{c_{0}}
    \]
    其中, \( R \) 为观察点 \( \boldsymbol{x} \) 与声源辐射声波时所在位置间的距离, \(\theta \) 为声源运动方向与声传播方向的夹角。
\end{enumerate}

\clearpage