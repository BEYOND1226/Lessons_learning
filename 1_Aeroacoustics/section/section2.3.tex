\begin{enumerate}
    \item 针对声学远场,证明近似表达式:
        \[ \frac{\partial^{2}}{\partial x_{i} \partial x_{j}}\left[\frac{T_{i j}(\boldsymbol{y})}{r}\right] \approx \frac{1}{c_{0}^{2}} \frac{\left(x_{i}-y_{i}\right)\left(x_{j}-y_{j}\right)}{r^{3}}\left[\frac{\partial^{2} T_{i j}(\boldsymbol{y})}{\partial \tau^{2}}\right]. \]

        根据偏分法则可以得到:
        \begin{equation}
            \begin{aligned}
                \frac{\partial^{2}}{\partial x_{i} \partial x_{j}}\left[\frac{T_{i j}(\boldsymbol{y})}{r}\right]
                &= \left[\frac{\partial^{2} T_{i j}(\mathbf{y})}{\partial \tau^{2}}\right] \frac{1}{r} \frac{\partial \tau}{\partial x_{i}} \frac{\partial \tau}{\partial x_{j}} \\
                &+ 2\left[\frac{\partial T_{i j}(\mathbf{y})}{\partial \tau}\right] \frac{\partial \tau}{\partial x_{i}} \frac{\partial(1 / r)}{\partial x_{i}}+\left[T_{i j}(\mathbf{y})\right] \frac{\partial(1 / r)}{\partial x_{i} \partial x_{j}}
            \end{aligned}
        \end{equation}
        对于声学远场,可以将忽略上式中的\(r^{-2}\)和\(r^{-3}\)项,因此有:
        \begin{equation}
            \label{eq:equation ignore r-2 r-3}
            \frac{\partial^{2}}{\partial x_{i} \partial x_{j}}\left[\frac{T_{i j}(\boldsymbol{y})}{r}\right]
            \approx \left[\frac{\partial^{2} T_{i j}(\mathbf{y})}{\partial \tau^{2}}\right] \frac{1}{r} \frac{\partial \tau}{\partial x_{i}} \frac{\partial \tau}{\partial x_{j}}
        \end{equation}
        其中,
        \begin{equation}
            \frac{\partial \tau}{\partial x_{i}}
            = \frac{\partial \tau}{\partial r} \frac{\partial r}{\partial x_{i}}
        \end{equation}
        根据\(\tau\)与\(r\)关系式:
        \begin{equation}
            \tau = t - \frac{r}{c_{0}}
        \end{equation}
        有:
        \begin{equation}
            \frac{\partial \tau}{\partial r}
            = - \frac{1}{c_{0}}
        \end{equation}
        又因为:
        \begin{equation}
            \begin{aligned}
                \frac{\partial r}{\partial x_{i}}
                &= \frac{\partial \sqrt{ \sum \left(x_{i}-y_{i}\right)^{2}}}{\partial x_{i}}
                &= \frac{x_{i} - y_{i}}{r}
            \end{aligned}
        \end{equation}
        因此有:
        \begin{equation}
            \frac{\partial \tau}{\partial x_{i}}
            = - \frac{1}{c_{0}} \frac{x_{i} - y_{i}}{r}
        \end{equation}
        同理:
        \begin{equation}
            \frac{\partial \tau}{\partial x_{j}}
            = - \frac{1}{c_{0}} \frac{x_{j} - y_{j}}{r}
        \end{equation}
        代入式\eqref{eq:equation ignore r-2 r-3},得:
        \begin{equation}
            \frac{\partial^{2}}{\partial x_{i} \partial x_{j}}\left[\frac{T_{i j}(\boldsymbol{y})}{r}\right] 
            \approx \frac{1}{c_{0}^{2}} \frac{\left(x_{i}-y_{i}\right)\left(x_{j}-y_{j}\right)}{r^{3}}\left[\frac{\partial^{2} T_{i j}(\boldsymbol{y})}{\partial \tau^{2}}\right]
        \end{equation}
        原式得证。

    \clearpage

    \item 对于等熵流动,
    \(\frac{\partial^{2}}{\partial \tau^{2}}\left(p^{\prime}-c_{0}^{2} \rho^{\prime}\right)=0\)
    一定成立吗?
    
        不一定。\( p^{\prime} = c_{0}^{2} \rho^{\prime} \)成立的前提是均匀介质,
        对于梯度较大的介质,\( p^{\prime} \neq  c_{0}^{2} \rho^{\prime} \),
        因此,\(\frac{\partial^{2}}{\partial \tau^{2}}\left(p^{\prime}-c_{0}^{2} \rho^{\prime}\right)=0\) 不一定成立。

        
    \item 参数\(p^{\prime}\)和\(\rho^{\prime}\)哪一个更适合描述非稳态低速燃烧流动产生的噪声?
        
        \(p^{\prime}\)更适合。
        非稳态低速燃烧流动涉及到能量方程,而参数\(p^{\prime}\)主要就源于能量方程,因此\(p^{\prime}\)更适合。
\end{enumerate}