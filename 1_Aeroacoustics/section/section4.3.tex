\section*{Section4.3}

\begin{enumerate}
    \item 将当地密度\( \rho \)、速度\(\boldsymbol{u}\)和压力\(p\)分解为时均值和脉动值两部分,即
    \begin{align*}
        \rho (\boldsymbol{x},t) &= \rho_{0} (\boldsymbol{x}) + \rho \prime (\boldsymbol{x},t) \\
        \boldsymbol{u} (\boldsymbol{x},t) &= \boldsymbol{u}_{0} (\boldsymbol{x}) + \boldsymbol{u} \prime (\boldsymbol{x},t) \\
        p (\boldsymbol{x},t) &= p_{0} (\boldsymbol{x}) + p \prime (\boldsymbol{x},t)
    \end{align*}
    对线性小振幅扰动,以\((\rho \prime , \rho_{0} \prime \boldsymbol{u} \prime, p \prime)\)为声学变量,建立线化欧拉方程组。

        对于连续方程:
        \begin{gather}
            \frac{\partial \rho}{\partial t}+\nabla \cdot(\rho \mathbf{u})=0
        \end{gather}
        代入声学变量,得:
        \begin{gather}
            \frac{\partial \rho^{\prime}}{\partial t}+\nabla \cdot\left(\rho_{0} \mathbf{u}_{0}+\rho^{\prime} \mathbf{u}_{0}+\rho_{0} \mathbf{u}^{\prime}+\rho^{\prime} \mathbf{u}^{\prime}\right)=0
        \end{gather}
        又因为:
        \begin{gather}
            \nabla \cdot\left(\rho_{0} \mathbf{u}_{0}\right) = - \frac{\partial \rho_{0}}{\partial t} = 0
        \end{gather}
        得到线化连续方程:
        \begin{equation}
            \begin{aligned}
                & \frac{\partial \rho^{\prime}}{\partial t}+\nabla \cdot\left(\rho^{\prime} \mathbf{u}_{0}+\rho_{0} \mathbf{u}^{\prime}+\rho^{\prime} \mathbf{u}^{\prime}\right) \\
                =& \frac{\partial \rho^{\prime}}{\partial t}+\mathbf{u}_{0} \cdot \nabla \rho^{\prime}+\rho^{\prime} \nabla \cdot \mathbf{u}_{0}+\rho_{0} \nabla \cdot \mathbf{u}^{\prime}+\mathbf{u}^{\prime} \cdot \nabla \rho_{0} \\
                =& -\nabla \cdot\left(\rho^{\prime} \mathbf{u}^{\prime}\right) 
            \end{aligned}        
        \end{equation}
        对于动量方程:
        \begin{gather}
            \rho \frac{\partial \mathbf{u}}{\partial t}+\rho \mathbf{u} \cdot \nabla \mathbf{u}+\nabla p=0
        \end{gather}
        代入声学变量,得:
        \begin{equation}
            \begin{aligned}
                & (\rho_{0} + \rho \prime) \frac{\partial (\boldsymbol{u}_{0}+ \boldsymbol{u} \prime) }{\partial t}+(\rho_{0} + \rho \prime) (\boldsymbol{u}_{0} + \boldsymbol{u} \prime ) \cdot \nabla (\boldsymbol{u}_{0} + \boldsymbol{u} \prime )+\nabla (p_{0}  + p \prime ) \\
                =& \rho_{0}\left(\frac{\partial \mathbf{u}^{\prime}}{\partial t}+\mathbf{u}_{0} \cdot \nabla \mathbf{u}^{\prime}\right)+\left(\rho_{0} \mathbf{u}_{0} \cdot \nabla \mathbf{u}_{0}+\nabla p_{0}\right)+\left(\rho_{0} \mathbf{u}^{\prime}+\rho^{\prime} \mathbf{u}_{0}\right) \cdot \nabla \mathbf{u}_{0}+\nabla p^{\prime} \\
                &+\left[\rho^{\prime}\left(\frac{\partial \mathbf{u}^{\prime}}{\partial t}+\mathbf{u}_{0} \cdot \nabla \mathbf{u}^{\prime}\right)+\rho \mathbf{u}^{\prime} \cdot \nabla \mathbf{u}^{\prime}+\rho^{\prime} \mathbf{u}^{\prime} \cdot \nabla \mathbf{u}_{0}\right]
                = 0
            \end{aligned}
        \end{equation}
        又因为:
        \begin{gather}
            \rho_{0} \mathbf{u}_{0} \cdot \nabla \mathbf{u}_{0}+\nabla p_{0}=0 \\
            \frac{\mathrm{D}_{0}}{\mathrm{D} t}=\frac{\partial}{\partial t}+\mathbf{u}_{0} \cdot \nabla
        \end{gather}
        得到线化动量方程:
        \begin{equation}
            \rho_{0} \frac{\mathrm{D}_{0} \mathbf{u}^{\prime}}{\mathrm{D} t}+\left(\rho_{0} \mathbf{u}^{\prime}+\rho^{\prime} \mathbf{u}_{0}\right) \cdot \nabla \mathbf{u}_{0}+\nabla p^{\prime}
            =-\rho^{\prime} \frac{\mathrm{D}_{0} \mathbf{u}^{\prime}}{\mathrm{D} t}-\rho_{0} \mathbf{u}^{\prime} \cdot \nabla \mathbf{u}^{\prime}-\rho^{\prime} \mathbf{u}^{\prime} \cdot \nabla \mathbf{u}_{0}
        \end{equation}
        对于能量方程:
        \begin{equation}
            \frac{\partial p}{\partial t}+\mathbf{u} \cdot \nabla p+\gamma p \nabla \cdot \mathbf{u}=0
        \end{equation}
        代入声学变量,得:
        \begin{equation}
            \begin{aligned}
                & \frac{\partial (p_{0}  + p \prime )}{\partial t}+\mathbf{(\boldsymbol{u}_{0}+ \boldsymbol{u} \prime)} \cdot \nabla (p_{0}  + p \prime )+\gamma (p_{0}  + p \prime ) \nabla \cdot \mathbf{(\boldsymbol{u}_{0}+ \boldsymbol{u} \prime)} \\
                =& \left(\frac{\partial p^{\prime}}{\partial t}+\mathbf{u}_{0} \cdot \nabla p^{\prime}\right)+\left(\mathbf{u}_{0} \cdot \nabla p_{0}+\gamma p_{0} \nabla \cdot \mathbf{u}_{0}\right)+\mathbf{u}^{\prime} \cdot \nabla p_{0}+\mathbf{u}^{\prime} \cdot \nabla p^{\prime} \\
                & +\gamma p_{0} \nabla \cdot \mathbf{u}^{\prime}+\gamma p^{\prime} \nabla \cdot \mathbf{u}^{\prime}+\gamma p^{\prime} \nabla \cdot \mathbf{u}_{0} \\
                =& 0
            \end{aligned}
        \end{equation}
        又因为:
        \begin{gather}
            \mathbf{u}_{0} \cdot \nabla p_{0}+\gamma p_{0} \nabla \cdot \mathbf{u}_{0}=0
        \end{gather}
        得到线化能量方程:
        \begin{equation}
            \frac{D_{0} p^{\prime}}{D t}+\mathbf{u}^{\prime} \cdot \nabla p_{0}+\gamma p_{0} \nabla \cdot \mathbf{u}^{\prime}+\gamma p^{\prime} \nabla \cdot \mathbf{u}_{0}
            =-\mathbf{u}^{\prime} \cdot \nabla p^{\prime}-\gamma p^{\prime} \nabla \cdot \mathbf{u}^{\prime}
        \end{equation}
        综上,得到线化欧拉方程组
        \begin{equation}
            \begin{gathered}
                \frac{\partial \rho^{\prime}}{\partial t}+\mathbf{u}_{0} \cdot \nabla \rho^{\prime}+\rho^{\prime} \nabla \cdot \mathbf{u}_{0}+\rho_{0} \nabla \cdot \mathbf{u}^{\prime}+\mathbf{u}^{\prime} \cdot \nabla \rho_{0}
                = -\nabla \cdot\left(\rho^{\prime} \mathbf{u}^{\prime}\right) 
                \\
                \rho_{0} \frac{\mathrm{D}_{0} \mathbf{u}^{\prime}}{\mathrm{D} t}+\left(\rho_{0} \mathbf{u}^{\prime}+\rho^{\prime} \mathbf{u}_{0}\right) \cdot \nabla \mathbf{u}_{0}+\nabla p^{\prime}
                =-\rho^{\prime} \frac{\mathrm{D}_{0} \mathbf{u}^{\prime}}{\mathrm{D} t}-\rho_{0} \mathbf{u}^{\prime} \cdot \nabla \mathbf{u}^{\prime}-\rho^{\prime} \mathbf{u}^{\prime} \cdot \nabla \mathbf{u}_{0}
                \\
                \frac{D_{0} p^{\prime}}{D t}+\mathbf{u}^{\prime} \cdot \nabla p_{0}+\gamma p_{0} \nabla \cdot \mathbf{u}^{\prime}+\gamma p^{\prime} \nabla \cdot \mathbf{u}_{0}
                =-\mathbf{u}^{\prime} \cdot \nabla p^{\prime}-\gamma p^{\prime} \nabla \cdot \mathbf{u}^{\prime}    
            \end{gathered}
        \end{equation}
\end{enumerate}