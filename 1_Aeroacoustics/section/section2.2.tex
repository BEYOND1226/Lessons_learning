\section*{Section2.2}

\begin{enumerate}
    \item 已知三维频域自由空间格林函数为
        $ G_{0} \left( \mathbf{x},\mathbf{y},\omega \right) = \frac{e^{i k r}}{4 \pi r} $ ,
        推导$ \frac{ \partial G_{0} }{ \partial y_{i} } $和
        $ \frac{ \partial^{2} G_{0} }{ \partial y_{i} \partial y_{j} } $的解析表达式。

        已知,在三维频域下:
        \begin{equation}
            r = \sqrt{ \sum^{n = 3}_{i = 1} \left( x_{i} - y_{i} \right)^{2}}
        \end{equation}
        因此有:
        \begin{equation}
            \begin{aligned}
                \frac{ \partial G_{0} }{ \partial y_{i} }
                &= \frac{ \partial }{ \partial r } \left(  \frac{e^{i k r}}{4 \pi r} \right) \frac{ \partial r }{ \partial y_{i} }\\
                &= \frac{i k r e^{ikr} - e^{ikr} }{4 \pi r^{2}} \frac{\partial \sqrt{ \sum^{n = 3}_{i = 1} \left( x_{i} - y_{i} \right)^{2}} }{\partial y_{i}} \\
                &= \left(\frac{i k}{4 \pi r} - \frac{1}{4 \pi r^{2}}\right) e^{ikr} \left(- \frac{x_{i} - y_{i}}{r}\right)\\
                &= \frac{x_{i} - y_{i}}{r} \left(\frac{e^{ikr}}{4 \pi r^{2}}-\frac{i k e^{ikr}}{4 \pi r}\right)
            \end{aligned}
        \end{equation}
        同理有:
        \begin{equation}
            \begin{aligned}
                \frac{ \partial^{2} G_{0} }{ \partial y_{i} \partial y_{j} }
                &= \frac{ \partial }{ \partial y_{j} } \left(\frac{ \partial G_{0} }{ \partial y_{i} } \right) \\
                &= \frac{ \partial }{ \partial r } \left(\frac{ \partial G_{0} }{ \partial y_{i} } \right) \frac{ \partial r }{ \partial y_{i} } \\
                &= \frac{x_{i}-y_{i}}{r^2} \left( - \frac{3 e^{ikr}}{4 \pi r^2} + \frac{3 i k e^{ikr}}{4 \pi r} + \frac{k^{2} e^{ikr}}{4 \pi }\right) \left(-\frac{x_{j}-y_{j}}{r}\right) \\
                &= \frac{(x_{i}-y_{i})(x_{j}-y_{j})}{r^3} \left(\frac{3 e^{ikr}}{4 \pi r^2} - \frac{3 i k e^{ikr}}{4 \pi  r} - \frac{k^{2} e^{ikr}}{4 \pi }\right)
            \end{aligned}
        \end{equation}
        综上,
        \begin{align}
            \frac{ \partial G_{0} }{ \partial y_{i} } 
            &= \frac{x_{i} - y_{i}}{r} \left(\frac{e^{ikr}}{4 \pi r^{2}}-\frac{i k e^{ikr}}{4 \pi r}\right) \\
            \frac{ \partial^{2} G_{0} }{ \partial y_{i} \partial y_{j} }
            &= \frac{(x_{i}-y_{i})(x_{j}-y_{j})}{r^3} \left(\frac{3 e^{ikr}}{4 \pi r^2} - \frac{3 i k e^{ikr}}{4 \pi r} - \frac{k^{2} e^{ikr}}{4 \pi }\right)
        \end{align}
    
    \clearpage

    \item 假设静止固体表面是可穿透的,并忽略粘性的贡献,写出Curle方程的频域积分公式。

        已知忽略粘性贡献的Curle方程为:
        \begin{equation}
            \begin{aligned}
                c_{0}^{2} \rho^{\prime}(\mathbf{x}, t) 
                &=\int_{V} \int_{-\infty}^{+\infty} T_{i j} \frac{\partial^{2} G}{\partial y_{i} \partial y_{j}} \mathrm{~d}^{3} \mathbf{y} \mathrm{d} \tau \\
                &-\int_{S} \int_{-\infty}^{+\infty}\left(\rho u_{i} u_{j}+p_{i j}\right) n_{j} \frac{\partial G}{\partial y_{i}} \mathrm{~d}^{2} \mathbf{y} \mathrm{d} \tau \\
                &-\int_{S} \int_{-\infty}^{+\infty} G \frac{\partial\left(\rho u_{j} n_{j}\right)}{\partial \tau} \mathrm{d}^{2} \mathbf{y} \mathrm{d} \tau
            \end{aligned}
        \end{equation}
        不妨设:
        \begin{align}
            F_{i}(\mathbf{y}, \tau)
            &=\left(\rho u_{i} u_{j}+p_{i j}\right) n_{j} \\
            Q(\mathbf{y}, \tau)
            &=\rho u_{j} n_{j}
        \end{align}
        代入自由格林函数$G_{0}$,根据$G_{0}$的性质:
        \begin{align}
            \frac{\partial G_{0}}{\partial y_{i}}
            &= -\frac{\partial G_{0}}{\partial x_{i}} \\
            \quad \frac{\partial^{2} G_{0}}{\partial y_{i} \partial y_{j}}
            &= \frac{\partial^{2} G_{0}}{\partial x_{i} \partial x_{j}}
        \end{align}
        可以得到:
        \begin{equation}
            \begin{aligned}
                c_{0}^{2} \rho^{\prime}(\mathbf{x}, t) 
                =&\frac{\partial^{2}}{\partial x_{i} \partial x_{j}} \int_{-\infty}^{+\infty} \int_{V} T_{i j}(\mathbf{y}, \tau) G_{0} \mathrm{~d}^{3} \mathbf{y} \mathrm{d} \tau \\
                &+\frac{\partial}{\partial x_{i}} \int_{-\infty}^{+\infty} \int_{S} F_{i}(\mathbf{y}, \tau) G_{0} \mathrm{~d}^{2} \mathbf{y d} \tau \\
                &-\int_{-\infty}^{+\infty} \int_{S} \frac{\partial Q(\mathbf{y}, \tau)}{\partial \tau} G_{0} \mathrm{~d}^{2} \mathbf{y} \mathrm{d} \tau \\
                =&\frac{\partial^{2}}{\partial x_{i} \partial x_{j}} \int_{V}\left[T_{i j}\left(\mathbf{y}, t-r / c_{0}\right)\right]_{\tau=t-r / c_{0}} \frac{\mathrm{d}^{3} \mathbf{y}}{4 \pi r} \\
                &+\frac{\partial}{\partial x_{i}} \int_{S}\left[F_{i}\left(\mathbf{y}, t-r / c_{0}\right)\right]_{\tau=t-r / c_{0}} \frac{\mathrm{d}^{2} \mathbf{y}}{4 \pi r} \\
                &-\int_{S}\left[\frac{\partial}{\partial \tau} Q\left(\mathbf{y}, t-r / c_{0}\right)\right]_{\tau=t-r / c_{0}} \frac{\mathrm{d}^{2} \mathbf{y}}{4 \pi r}
            \end{aligned}
        \end{equation}
        根据Fourier变换,可得:
        \begin{equation}
            \begin{aligned}
                \left( c_{0}^{2} \tilde{\rho^{\prime}}(\mathbf{x}, \omega) \right)_{quadrupole}
                &= \frac{\partial^{2}}{\partial x_{i} \partial x_{j}} \int_{V} \int_{-\infty}^{+\infty} T_{i j}(\mathbf{y}, t-r / c_{0}) e^{i \omega t}\mathrm{d} t \frac{\mathrm{d}^{3} \mathbf{y}}{4 \pi r} \\
                &= \frac{\partial^{2}}{\partial x_{i} \partial x_{j}} \int_{V} \widetilde{T_{i j}}(\mathbf{y}, \omega) e^{i \omega r / c_{0}} \frac{\mathrm{d}^{3} \mathbf{y}}{4 \pi r}
            \end{aligned}
        \end{equation}
        同理有:
        \begin{equation}
            \begin{aligned}
                \left( c_{0}^{2} \tilde{\rho^{\prime}}(\mathbf{x}, \omega) \right)_{dipole}
                &= \frac{\partial}{\partial x_{i}} \int_{S} \int_{-\infty}^{+\infty} F_{i}(\mathbf{y}, t-r / c_{0}) e^{i \omega t}\mathrm{d} t \frac{\mathrm{d}^{2} \mathbf{y}}{4 \pi r} \\
                &= \frac{\partial}{\partial x_{i}} \int_{S} \widetilde{F_{i}}(\mathbf{y}, \omega) e^{i \omega r / c_{0}} \frac{\mathrm{d}^{2} \mathbf{y}}{4 \pi r}
            \end{aligned}
        \end{equation}
        根据Fourier变换的偏分性质,可得:
        \begin{equation}
            \begin{aligned}
                \left( c_{0}^{2} \tilde{\rho^{\prime}}(\mathbf{x}, \omega) \right)_{monopole}
                &= \int_{S} \int_{-\infty}^{+\infty} \frac{\partial}{\partial \tau} \left[Q\left(\mathbf{y}, t-r / c_{0}\right)\right] e^{i \omega t}\mathrm{d} t \frac{\mathrm{d}^{2} \mathbf{y}}{4 \pi r} \\
                &= \int_{S} - i \omega \widetilde{Q}(\mathbf{y}, \omega) e^{i \omega r / c_{0}} \frac{\mathrm{d}^{2} \mathbf{y}}{4 \pi r}
            \end{aligned}
        \end{equation}
        综上,curle方程的频域积分表达式为:
        \begin{equation}
            \begin{aligned}
                c_{0}^{2} \tilde{\rho^{\prime}}(\mathbf{x}, \omega)
                &= \frac{\partial^{2}}{\partial x_{i} \partial x_{j}} \int_{V} \widetilde{T_{i j}}(\mathbf{y}, \omega) e^{i \omega r / c_{0}} \frac{\mathrm{d}^{3} \mathbf{y}}{4 \pi r} \\
                &+ \frac{\partial}{\partial x_{i}} \int_{S} \widetilde{F_{i}}(\mathbf{y}, \omega) e^{i \omega r / c_{0}} \frac{\mathrm{d}^{2} \mathbf{y}}{4 \pi r} \\
                &+ \int_{S} i \omega \widetilde{Q}(\mathbf{y}, \omega) e^{i \omega r / c_{0}} \frac{\mathrm{d}^{2} \mathbf{y}}{4 \pi r}
            \end{aligned}
        \end{equation}
\end{enumerate}

\clearpage