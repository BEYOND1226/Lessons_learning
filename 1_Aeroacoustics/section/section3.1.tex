\section*{Section3.1}

\begin{enumerate}
    \item 对无黏、均熵可压缩流动, 证明涡运动方程的表达形式可写为
    \[
    \frac{D}{D t}\left(\frac{\boldsymbol{\omega}}{\rho}\right)
    =\frac{\boldsymbol{\omega}}{\rho} \cdot \nabla \mathbf{u}
    \]

        已知:
        \begin{equation}
            \frac{\partial \mathbf{u}}{\partial t}+\nabla H+\boldsymbol{\omega} \times \mathbf{u}-T \nabla s-\mathbf{e}=0
        \end{equation}
        两边求旋度得:
        \begin{equation}
            \nabla \times \left(\frac{\partial \mathbf{u}}{\partial t}+\nabla H+\boldsymbol{\omega} \times \mathbf{u}-T \nabla s-\mathbf{e}\right)=0
        \end{equation}
        因为\( \nabla \times \nabla H \equiv 0 \),因此有:
        \begin{equation}
            \frac{\partial \boldsymbol{\omega}}{\partial t}+\nabla \times(\boldsymbol{\omega} \times \mathbf{u})-\nabla T \times \nabla s-\nabla \times \mathbf{e}=0
        \end{equation}
        又因为:
        \begin{equation}
            \begin{aligned}
                \nabla \times(\boldsymbol{\omega} \times \mathbf{u})
                &= \boldsymbol{\omega}(\nabla \cdot \mathbf{u})-\mathbf{u}(\nabla \cdot \boldsymbol{\omega})+(\mathbf{u} \cdot \nabla) \boldsymbol{\omega}-(\boldsymbol{\omega} \cdot \nabla) \mathbf{u} \\
                \nabla \times \mathbf{e}
                &= \nu \nabla^{2} \boldsymbol{\omega}
            \end{aligned}
        \end{equation}
        结合\(\nabla \cdot \boldsymbol{\omega} = \nabla \cdot (\nabla \times \mathbf{u}) = 0\)得:
        \begin{equation}
            \begin{aligned}
                & \frac{\partial \boldsymbol{\omega}}{\partial t}+\nabla \times(\boldsymbol{\omega} \times \mathbf{u})-\nabla T \times \nabla s-\nabla \times \mathbf{e} \\
                =& \frac{\partial \boldsymbol{\omega}}{\partial t} + (\mathbf{u} \cdot \nabla) \boldsymbol{\omega} 
                + \boldsymbol{\omega}(\nabla \cdot \mathbf{u}) - (\boldsymbol{\omega} \cdot \nabla) \mathbf{u} \\
                &- \mathbf{u}(\nabla \cdot \boldsymbol{\omega})
                - \nabla T \times \nabla s
                - \nu \nabla^{2} \boldsymbol{\omega} \\
                =& \frac{D \boldsymbol{\omega}}{D t}
                + \boldsymbol{\omega}(\nabla \cdot \mathbf{u})
                - (\boldsymbol{\omega} \cdot \nabla) \mathbf{u}
                - \nabla T \times \nabla s
                - \nu \nabla^{2} \boldsymbol{\omega} \\
                =& 0 
            \end{aligned}
        \end{equation}
        根据连续方程,有:
        \begin{equation}
            \begin{aligned}
                \frac{\boldsymbol{\omega}}{\rho^{2}} ( \frac{D \rho }{D t} + \rho ( \nabla \cdot \mathbf{u} ) )
                = \frac{\boldsymbol{\omega}}{\rho^{2}} \frac{D \rho }{D t} + \frac{\boldsymbol{\omega}}{\rho} ( \nabla \cdot \mathbf{u} ) 
                = 0
            \end{aligned}
        \end{equation}
        因此有:
        \begin{equation}
            \begin{aligned}
                &\frac{1}{\rho} \frac{D \boldsymbol{\omega}}{D t}
                - (\frac{\boldsymbol{\omega}}{\rho} \cdot \nabla) \mathbf{u}
                - \frac{1}{\rho} \nabla T \times \nabla s
                - \frac{\nu}{\rho} \nabla^{2} \boldsymbol{\omega}
                - \frac{\boldsymbol{\omega}}{\rho^{2}} \frac{D \rho }{D t} \\
                =& \frac{D}{D t}\left(\frac{\boldsymbol{\omega}}{\rho}\right)-\left(\frac{\boldsymbol{\omega}}{\rho} \cdot \nabla\right) \mathbf{u}-\frac{1}{\rho} \nabla T \times \nabla s-\frac{\nu}{\rho} \nabla^{2} \boldsymbol{\omega} \\
                =& 0
            \end{aligned}
        \end{equation}
        
        对于无粘、等熵流动,有:
        \begin{align}
            \frac{1}{\rho} \nabla T \times \nabla s = 0 \\
            \frac{\nu}{\rho} \nabla^{2} \boldsymbol{\omega} = 0
        \end{align}
        因此有:
        \begin{equation}
            \begin{aligned}
                \frac{D}{D t}\left(\frac{\boldsymbol{\omega}}{\rho}\right)
                = \left(\frac{\boldsymbol{\omega}}{\rho} \cdot \nabla\right) \mathbf{u}
                = \frac{\boldsymbol{\omega}}{\rho} \cdot \nabla \mathbf{u}
            \end{aligned}
        \end{equation}
        原式得证。


    \item 对无黏正压流体的可压缩运动, 证明浴运动方程的表达形式可写为
    \[
    \frac{D \boldsymbol{\omega}}{D t}
    =(\boldsymbol{\omega} \cdot \nabla) \mathbf{u}-\boldsymbol{\omega}(\nabla \cdot \mathbf{u})
    \]

        根据第1题中的推导,已知:
        \begin{equation}
            \begin{aligned}
                \frac{D \boldsymbol{\omega}}{D t}
                + \boldsymbol{\omega}(\nabla \cdot \mathbf{u})
                - (\boldsymbol{\omega} \cdot \nabla) \mathbf{u}
                - \nabla T \times \nabla s
                - \nu \nabla^{2} \boldsymbol{\omega}
                = 0
            \end{aligned}
        \end{equation}
        对于无粘流动,有:
        \begin{align}
            \frac{\nu}{\rho} \nabla^{2} \boldsymbol{\omega} = 0
        \end{align}
        对于正压流动,有:
        \begin{align}
            \nabla T \times \nabla s = 0
        \end{align}
        因此有:
        \begin{equation}
            \frac{D \boldsymbol{\omega}}{D t}
            + \boldsymbol{\omega}(\nabla \cdot \mathbf{u})
            - (\boldsymbol{\omega} \cdot \nabla) \mathbf{u}
            = 0
        \end{equation}
        \begin{equation}
            \frac{D \boldsymbol{\omega}}{D t}
            =(\boldsymbol{\omega} \cdot \nabla) \mathbf{u}-\boldsymbol{\omega}(\nabla \cdot \mathbf{u})        
        \end{equation}
        原式得证。
\end{enumerate}

\clearpage