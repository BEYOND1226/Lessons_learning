\documentclass[12pt,a4paper]{article}

%---------------------------------设置字体大小---------------------------------%
\usepackage{type1cm}
% 字号与行距,统一前缀s(a.k.a size)
\newcommand{\sChuhao}{\fontsize{42pt}{63pt}\selectfont}                 % 初号, 1.5倍
\newcommand{\sYihao}{\fontsize{26pt}{36pt}\selectfont}                  % 一号, 1.4倍
\newcommand{\sErhao}{\fontsize{22pt}{28pt}\selectfont}                  % 二号, 1.25倍
\newcommand{\sXiaoer}{\fontsize{18pt}{18pt}\selectfont}                 % 小二, 单倍
\newcommand{\sSanhao}{\fontsize{16pt}{24pt}\selectfont}                 % 三号, 1.5倍
\newcommand{\sXiaosan}{\fontsize{15pt}{22pt}\selectfont}                % 小三, 1.5倍
\newcommand{\sSihao}{\fontsize{14pt}{21pt}\selectfont}                  % 四号, 1.5倍
\newcommand{\sHalfXiaosi}{\fontsize{12.5pt}{16.25pt}\selectfont}        % 半小四, 1.25倍
\newcommand{\sLargeHalfXiaosi}{\fontsize{13pt}{19pt}\selectfont}        % 半小四, 1.5倍
\newcommand{\sXiaosi}{\fontsize{12pt}{14.4pt}\selectfont}               % 小四, 1.25倍
\newcommand{\sLargeWuhao}{\fontsize{11pt}{11pt}\selectfont}             % 大五, 单倍
\newcommand{\sWuhao}{\fontsize{10.5pt}{10.5pt}\selectfont}              % 五号, 单倍
\newcommand{\sXiaowu}{\fontsize{9pt}{9pt}\selectfont}                   % 小五, 单倍

%---------------------------------设置中文字体---------------------------------%
\usepackage{fontspec}
\usepackage[SlantFont,BoldFont,CJKchecksingle]{xeCJK}
\usepackage{CJKnumb}
% 使用 Adobe 字体
\newcommand\adobeSog{SimSun}
\newcommand\adobeHei{SimHei}
\newcommand\adobeKai{KaiTi}
\newcommand\adobeFag{FangSong}
\newcommand\codeFont{Consolas}
% 设置字体
\defaultfontfeatures{Mapping=tex-text}
\setCJKmainfont[ItalicFont=\adobeKai, BoldFont=\adobeHei]{\adobeSog}
\setCJKsansfont[ItalicFont=\adobeKai, BoldFont=\adobeHei]{\adobeSog}
\setCJKmonofont{\codeFont}
\setmonofont{\codeFont}
% 设置字体族
\setCJKfamilyfont{song}{\adobeSog}      % 宋体
\setCJKfamilyfont{hei}{\adobeHei}       % 黑体
\setCJKfamilyfont{kai}{\adobeKai}       % 楷体
\setCJKfamilyfont{fang}{\adobeFag}      % 仿宋体

% 新建字体命令,统一前缀f(a.k.a font)
\newcommand{\fSong}{\CJKfamily{song}}
\newcommand{\fHei}{\CJKfamily{hei}}
\newcommand{\fFang}{\CJKfamily{fang}}
\newcommand{\fKai}{\CJKfamily{kai}}

%--------------------------设置中文段落缩进与正文版式--------------------------%
\XeTeXlinebreaklocale "zh"                      % 使用中文的换行风格
\XeTeXlinebreakskip = 0pt plus 1pt              % 调整换行逻辑的弹性大小
\usepackage{indentfirst}                        % 段首空格设置
\setlength{\parindent}{26pt}                    % 段首空格长度
\setlength{\parskip}{3pt plus 1pt minus 1pt}    % 段落间距
\renewcommand{\baselinestretch}{1.25}           % 行距

%---------------------------------纸张大小设置---------------------------------%
\usepackage{geometry}
% 普通A4格式缩进
% \geometry{left=2.5cm,right=2.5cm,top=2.5cm,bottom=2.5cm}
% 论文标准缩进
\geometry{left=1.25in,right=1.25in,top=1in,bottom=1.5in}

% 公式对齐
\usepackage{amsmath}

\usepackage{enumerate}
\usepackage{enumitem}
\setlist[enumerate,1]{label=\arabic*.,font=\textup,
leftmargin=7mm,labelsep=1.5mm,topsep=0mm,itemsep=-0.8mm}
\setlist[enumerate,2]{label=(\arabic*),font=\textup, 
leftmargin=7mm,labelsep=1.5mm,topsep=-0.8mm,itemsep=-0.8mm}

\renewcommand{\baselinestretch}{1.25}
\sHalfXiaosi\fSong

%----------------------------设置段落标题与目录格式----------------------------%
\usepackage[sf]{titlesec}
\usepackage{titletoc}
\usepackage{caption}

\titleformat{\chapter}[hang]{\normalfont\sSanhao\filcenter\fHei\bf}%
{\sSanhao{\chaptertitlename}}{20pt}{\sSanhao}
\titleformat{\section}[hang]{\fHei \bf \sSihao}% 四号
{\sSihao \thesection}{0.5em}{}{}
\titleformat{\subsection}[hang]{\fHei \bf \sXiaosi}% 小四 
{\sXiaosi \thesubsection}{0.5em}{}{}
\titleformat{\subsubsection}[hang]{\fHei \bf}%
{(\arabic{subsubsection})}{0.5em}{}{}   % 小标题式的subsubsection:(4) 标题

% 缩小正文中各级标题之间的缩进
\titlespacing{\chapter}{0pt}{-3ex plus .1ex minus .2ex}{0.25em}
\titlespacing{\section}{0pt}{-0.2em}{0em}
\titlespacing{\subsection}{0pt}{0.5em}{0em}
\titlespacing{\subsubsection}{0pt}{0.25em}{0pt}

%------------------------------添加插图与表格控制------------------------------%
\usepackage{graphicx}
\usepackage[font=small,labelsep=quad]{caption}
\usepackage{wrapfig}
\usepackage{multirow,makecell}
\usepackage{longtable}
\usepackage{booktabs}
\usepackage{tabularx}
\usepackage{setspace}
\captionsetup[table]{labelfont=bf,textfont=bf}